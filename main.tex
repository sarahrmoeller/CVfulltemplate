            %%%%%%%%%%%%%%%%%%%%%%%%%%%%%%%%%%%%%%%%%%%%%%%%%%%%%%%%%%%%%%%%%%%%%%
% LaTeX Template: Curriculum Vitae
%
% Source: http://www.howtotex.com/
% Feel free to distribute this template, but please keep the
% referal to HowToTeX.com.
% Date: July 2011
% 
%%%%%%%%%%%%%%%%%%%%%%%%%%%%%%%%%%%%%%%%%%%%%%%%%%%%%%%%%%%%%%%%%%%%%%
% How to use writeLaTeX: 
%
% You edit the source code here on the left, and the preview on the
% right shows you the result within a few seconds.
%
% Bookmark this page and share the URL with your co-authors. They can
% edit at the same time!
%
% You can upload figures, bibliographies, custom classes and
% styles using the files menu.
%
% If you're new to LaTeX, the wikibook is a great place to start:
% http://en.wikibooks.org/wiki/LaTeX
%
%%%%%%%%%%%%%%%%%%%%%%%%%%%%%%%%%%%%%%%%%%%%%%%%%%%%%%%%%%%%%%%%%%%%%%
\documentclass[paper=a4,fontsize=11pt]{article} % KOMA-article class

\usepackage{natbib}
\usepackage{bibentry}
\usepackage[english]{babel}
\usepackage[utf8x]{inputenc}
\usepackage[protrusion=true,expansion=true]{microtype}
\usepackage{amsmath,amsfonts,amsthm}     % Math packages
\usepackage{graphicx}                    % Enable pdflatex
\usepackage[svgnames]{xcolor}            % Colors by their 'svgnames'
\usepackage[margin=3cm]{geometry}
	%\textheight=700px                    % Saving trees ;-)
\usepackage{url}
\usepackage{hyperref}
\hypersetup{
    colorlinks=true,
    linkcolor=blue,
    filecolor=magenta,      
    urlcolor=blue,
}

\usepackage{hanging}
\usepackage{fancyhdr}
\frenchspacing              % Better looking spacings after periods
%\pagestyle{empty}           % No pagenumbers/headers/footers
    \pagestyle{fancy}
    \chead{Moeller CV 2020 page \thepage}
%%% Custom sectioning (sectsty package)
%%% ------------------------------------------------------------
\usepackage{sectsty}

\sectionfont{%			            % Change font of \section command
	\usefont{OT1}{phv}{b}{n}%		% bch-b-n: CharterBT-Bold font
	\sectionrule{0pt}{0pt}{-5pt}{3pt}}

%%% Macros
%%% ------------------------------------------------------------
\newlength{\spacebox}
\settowidth{\spacebox}{8888888888}			% Box to align text
\newcommand{\sepspace}{\vspace*{1em}}		% Vertical space macro

\newcommand{\MyName}[1]{ % Name
		\Huge \usefont{OT1}{phv}{b}{n} \hfill #1
		\par \normalsize \normalfont}
		
\newcommand{\MySlogan}[1]{ % Slogan (optional)
		\large \usefont{OT1}{phv}{m}{n}\hfill \textit{#1}
		\par \normalsize \normalfont}

\newcommand{\NewPart}[1]{\section*{\uppercase{#1}}}

\newcommand{\PersonalEntry}[2]{
		\hangindent=2em\hangafter=0 % Indentation
		\hspace*{\fill}
		\parbox{\spacebox}{        % Box to align text
		\textit{#1}}		       % Entry name (birth, address, etc.)
		\hspace{1.5em} #2 \par}    % Entry value
		
\newcommand{\SkillsEntry}[2]{      % Same as \PersonalEntry
		\noindent\hangindent=2em\hangafter=0 % Indentation
		\parbox{\spacebox}{        % Box to align text
		\textit{#1}}			   % Entry name (birth, address, etc.)
		\hspace{1.5em} #2 \par}    % Entry value	

\newcommand{\ListEntry}[3]{
		\noindent
		\parbox{13em}{        % Box to align text
            #1}
        \hspace{.5em} 
        \parbox{23em}{
            #2} \hfill
		\colorbox{White}{%
			\parbox{5em}{%
			\hfill\color{Black}#3}}  \par}
			
\newcommand{\CourseEntry}[3]{
		\noindent
		\parbox{19em}{#1}
        \hspace{.5em} 
        \parbox{18.75em}{#2} 
        \hfill
		\colorbox{White}{%
			\parbox{4.5em}{%
			\hfill\color{Black}#3}}  \par}
		
		
\newcommand{\EducationEntry}[4]{
		\noindent \textbf{#1} \hfill      % Study
		\colorbox{White}{%
			\parbox{6em}{%
			\hfill\color{Black}#2}} \par  % Duration
		\noindent \textit{#3} \par        % School
		\noindent\hangindent=2em\hangafter=0 \small #4 % Description
		\normalsize \par}

\newcommand{\WorkEntry}[4]{				  % Same as \EducationEntry
		\noindent \textbf{#1} \hfill      % Jobname
		\colorbox{White}{%
			\parbox{6em}{%
			\hfill\color{Black}#2}} \par  % Duration
		\noindent \textit{#3} \par              % Company
		\noindent\hangindent=2em\hangafter=0 \small #4 % Description
		\normalsize \par}
		
\newcommand{\PresentationEntry}[1]{
		\noindent\hangindent=0em\hangafter=0 % Indentation
		\hspace*{}
		\parbox{\spacebox}{        % Box to align text
	    }		      
		\hspace{} #1 \par}    % Entry value

%%% Begin Document
%%% ------------------------------------------------------------
\begin{document}
\thispagestyle{empty}
% you can upload a photo and include it here...
%\begin{wrapfigure}{l}{0.5\textwidth}
%	\vspace*{-2em}
%		\includegraphics[width=0.15\textwidth]{photo}
%\end{wrapfigure}

\MyName{Sarah Moeller}
%\MySlogan{Curriculum Vitae}

\sepspace

%%% Personal details
%%% ------------------------------------------------------------
%\NewPart{Personal details}{}

\PersonalEntry{}{Department of Linguistics}
\PersonalEntry{}{Institute of Cognitive Science}
\PersonalEntry{}{Hellems 290, 295 UCB, Boulder, CO}
\PersonalEntry{}{\url{sarah.moeller@colorado.edu}}
\PersonalEntry{}{\url{linkedin.com/in/sarahrmoeller}}
\PersonalEntry{}{\url{github.com/theSarahRu}}
%\PersonalEntry{}{(636) 579-3048}
\sepspace

%%% Education
%%% ------------------------------------------------------------
\NewPart{Education}{}

\EducationEntry{Ph.D. Linguistics and Cognitive Science}{May 2021}{University of Colorado Boulder}{Dissertation: ``Integrating Machine Learning into Language Documentation and Description''}
\sepspace

%\EducationEntry{College Teaching Certificate}{2021}{University of Colorado Boulder}{}
%\vspace{}

\EducationEntry{M.A. Linguistics}{2020}{University of Colorado Boulder}
\vspace{}

\EducationEntry{M.A. Applied Linguistics}{2010}{Dallas International University}
\vspace{}

\EducationEntry{B.A. History}{2002}{Thomas Edison State University}
%\sepspace
%\vspace{}

%%% Research experience
%%% ------------------------------------------------------------
\NewPart{Research Experience}{}

\WorkEntry{Integrating Machine Learning into Language Documentation and Description}{2016-present}{University of Colorado}{Research question: how does early analytic decisions by linguists inform NLP models and how should integration of machine learning influence linguists' workflow? I devised new NLP methods for morphological analysis and human-in-the-loop descriptive work. Tested assumptions about the importance of part-of-speech (POS) tagging. Used limited and noisy field data from varying sources. 
%I experimented with machine learning for morphological analysis and description, training. 
Publications in 
\textit{Workshop on Polysynthetic Languages}, \textit{ComputEL}, \textit{EMNLP}.
%\textit{NAACL}.
}
\sepspace

\WorkEntry{Universal Meaning Representations}{2019-2020}{Computational Language and Education Research (CLEAR)}{Project aim: construct Russian Abstract Meaning Representations (AMR) that cohere with new cross-linguistic guidelines for improving natural language understanding. I built a proposition bank of semantic frames for %98
highly-frequent Russian predicates and created a gold standard corpus for automated English-to-Russian projection of semantic roles. Publication in \textit{LREC}.}
\sepspace

\WorkEntry{``Teacher-in-the-Loop''}{2018}{Language Technology for Language Documentation and Revitalization Workshop}{Designed software to assist language teachers searching small, unannotated corpora for example of specific linguistic phenomena. Publication in \textit{STLU-CCRL}}
\sepspace

\WorkEntry{Automated Transcription of Low-Resource Languages}{2018}{Language Software Development (LSDev), SIL International}{Goal: build software to assist transcription of languages that have no established orthography. I initiated research on how to integrate speech recognition for low-resource languages and on principles of interface design for users with low computer literacy. The software is under development.}
\sepspace

\WorkEntry{Supersense Disambiguation of Prepositions}{2017-2018}{Computational Language and Education Research (CLEAR)}{Focus: data-driven semantic disambiguation of prepositions and possessives. The project introduced the Semantic Network of Adposition and Case Supersense (SNACS) hierarchy and created an annotated corpus. I co-developed hierarchy and led the English annotation team. Publication in \textit{ACL}.}
\sepspace

\WorkEntry{VerbNet}{2017-2018}{Computational Language and Education Research (CLEAR)}{In order to improve VerbNet's interoperability and accuracy, I redesigned its syntax representation based on accurate linguistic knowledge. The new representation allows direct mapping between VerbNet to Universal Dependencies (UD).}
\sepspace

\WorkEntry{Language Documentation and Community Development}{2013-2015}{In cooperation with Ufuq-S and SIL International}{I conducted documentary fieldwork on two threatened languages spoken in Azerbaijan, curating endangered data and surveying endangered linguistic and cultural practices. I preserved two endangered corpora in long-term digital repositories.}
\sepspace
%\vspace{}

\WorkEntry{Metaphor Project}{2011-2013}{Language Computer Corporation}{A DARPA-funded project to automatically detect metaphors in Russian, Spanish, Farsi, and English. I developed annotation guidelines for defining and analyzing metaphors, mapped Russian dependency grammar to Stanford dependencies, and employed linguistic knowledge to improve Russian and Farsi syntactic parsers. These contributions led to the open access ``Corpus of Rich Metaphor Annotation" at ISI USC and a co-authored publication in \textit{CoNLL}. I also contributed an analysis regarding the feasibility of natural language understanding of author's intent.
}
\sepspace

\WorkEntry{Service Learning}{2010}{Dallas International University and Mississippi Band of Choctaw Indians}{I initiated and led a 3-day student service-learning project to identify and preserve legacy language data. Much data that had been lost in old computers was recovered and made available again.}
\sepspace

\WorkEntry{Student Internship}{2009-2010}{International Linguistics Office, SIL International}
{My responsibility was to digitize and train linguists to use a digitization lab and assist them in preparing legacy language data for archiving, resulting in two completed corpora and a digitization handbook.}
%\sepspace
\vspace{}
%\newpage

%%% Work experience
%%% ------------------------------------------------------------
\NewPart{Teaching experience}{}

%\WorkEntry{Research Assistant}{2017-present}{CLEAR, University of Colorado Boulder}{}
%\vspace{}

\WorkEntry{Graduate Part-time Instructor}{2020}{Department of Linguistics, University of Colorado Boulder}{}
\vspace{}

\WorkEntry{Teaching Assistant}{2016-2019}{Department of Linguistics, University of Colorado Boulder}{}
\vspace{}

%\WorkEntry{Project Manager Intern}{2018}{Language Software Development (LSDev), SIL International}{}
%\vspace{}

%\WorkEntry{Linguist}{2011-2015}{Language Computer Corporation}{}
%\vspace{}

\WorkEntry{Adjunct Instructor}{2011-2016}{Dallas International University}{}
\vspace{}

\WorkEntry{Teaching Assistant}{2010}{Dallas International University}{}
\vspace{}

\WorkEntry{English Teacher and Russian-English Interpreter}{2002-2008}{free-lance}{}
\vspace{}

%\WorkEntry{Job name}{2010-2011}{Company Name inc., Part-time}{Job description goes here. To maintain a stylish look, try to fill this description with a few lines of text. Do the same for the other entries in this section.}

%%% Prof service
%%% ------------------------------------------------------------
\NewPart{Professional Service}{}

%\WorkEntry{Secretary}{2020-present}{ACL Special Interest Group on Endangered Languages}{}
%\sepspace
%\vspace{}

\WorkEntry{Editorial Board}{2018-present}{Colorado Research in Linguistics (CRiL)}{}
%\sepspace
\vspace{}

\WorkEntry{Program Chair and Track Chair}{2021}{Fourth Workshop on Computational Methods for Endangered Languages (ComputEL-4)}{}
%\sepspace
\vspace{}

\WorkEntry{Ad hoc reviewer}{2020}{Empirical Methods in Natural Language Processing (EMNLP)}{}
%\sepspace
\vspace{}

\WorkEntry{Ad hoc reviewer}{2019, 2020}{Language Resources and Evaluation Conference (LREC)}{}
%\sepspace
\vspace{}

\WorkEntry{Program Committee}{2019, 2020}{International Workshop on Designing Meaning Representations (DMR)}{}
%\sepspace
\vspace{}

\WorkEntry{Student Representative}{2019}{Dept. of Linguistics Hiring Committee}{}
%\sepspace
\vspace{}

\WorkEntry{Ad hoc reviewer}{2019}{Language Documentation \& Conservation (LD&C)}{}
%\sepspace
\vspace{}

\WorkEntry{Student Representative}{2018}{Dept. of Linguistics Faculty}{}
%\sepspace
\vspace{}

\WorkEntry{Chairman}{2010}{DIU Student Body Association}{}
%\sepspace
\vspace{}

\WorkEntry{Secretary}{2009-2010}{DIU Student Body Association}{}
%\sepspace
\vspace{}

%\WorkEntry{English as Second Language Teacher}{multiple}{Community classes for refugees and immigrants \\St. Louis, MO 2000-2001; Dallas, TX 2010-2013}
%\sepspace
%\vspace{}

%\WorkEntry{Volunteer}{1997-2000}{MC ATIA Children's Home, Moscow, Russia}
%\sepspace
%\vspace{}

%%% Mentorship 
%%% ------------------------------------------------------------

\NewPart{Mentoring Experience}{}
%\NewPart{Mentoring/Management Experience}

\WorkEntry{Supervisor}{2018-present}{Computational Language and Education Research (CLEAR)}{Training annotators and manage research assistants in various projects; setting deadlines and accountability protocols; maintaining clear communication between them and the principal investigators.}
\sepspace

%\WorkEntry{Peer Mentor}{2016-present}{University of Colorado Boulder}{Met with and advise 1-2 first-year MA graduate students who are interested in pursuing a PhD.}
%\sepspace

\WorkEntry{Principal Investigator}{2019-2020}{Dissertation Research}{Under a CARTSS grant I hired two undergraduate and one MA research assistant to preprocess language data from multiple formats. I trained the computer science undergraduates in basic linguistic analysis and guided the linguistics MA student who was wrote Python script to ``clean up'' an Arapaho corpus. The MA student became a co-author on a
\textit{EMNLP} 
publication and has continued research on Arapaho.}
\sepspace

\WorkEntry{Project Manager Internship}{2018}{LSDev, SIL International}{Coordinated communication and clarified goals among clients/users, software developers, and interface designer. I established and maintain civil and productive conversations in face of conflicting interests.}
\sepspace

\WorkEntry{Project Lead/Linguist}{2011-2013}{Language Computer Corporation}{Hired and trained annotators in linguistics and metaphor theory, developed annotation guidelines, and managed four annotation teams.}
\sepspace

\WorkEntry{English Teacher /& English-Russian Interpreter}{2002-2008}{Free-lance}{Created and managed my own business.}

%%% Prof memberships
%%% ------------------------------------------------------------

%\NewPart{Professional Associations}{}

%\SkillsEntry{Member}{Linguistic Society of America (LSA)}
%\SkillsEntry{Member}{Association of Computational Linguistics (ACL)}
%\SkillsEntry{Secretary}{ACL Special Interest Group for Endangered Languages}


%%% Certifications and Prof Dev
%%% ------------------------------------------------------------

%\NewPart{Professional Development and Association}{}

%\ListEntry{Certificate Level 3}{Test of Russian as a Foreign Language}{}
%\ListEntry{Russian Linguistics Graduate Courses}{Lomonosov Moscow State University}{2002}
%\ListEntry{}{Lisbon Machine Learning School (LxMLS)}{} % {2018}

%%% Skills
%%% ------------------------------------------------------------
\NewPart{Skills}{}

\SkillsEntry{Languages}{English (mother tongue)}
\SkillsEntry{}{Russian (fluent)}
\SkillsEntry{}{German (conversant)}
\SkillsEntry{}{Georgian, French, Latin, Koine Greek (beginner)}

\SkillsEntry{Technical}{Python (scikit-learn, Keras), R, regex, 
%\LaTeX, 
Figma. Familiar with: Java, C, SQL}

%%% Grants and Awards
%%% ------------------------------------------------------------

\NewPart{Grants and Awards}{}

\ListEntry{Individual Research Grant}{International Literacy and Development (ILAD)}{2020}
\ListEntry{Graduate Student Grant}{Center for Advancement of Teaching in Social Sciences (CARTSS)}{2019}
\ListEntry{Summer Research Fellowship}{Dept. of Linguistics, U. of CO Boulder}{2017, 2018}
\ListEntry{Department Fellowship}{Dept. of Linguistics, U. of CO Boulder}{2016-2017}

%%% Publications
%%% ------------------------------------------------------------

\bibliographystyle{unsrtnat}
\renewcommand{\bibsection}{\section*{PUBLICATIONS}}
\bibliography{publication.bib}
\bibentry{moeller_emnlp_2020}
\bibentry{neubig_summary_2020}
\bibentry{moeller_russian_2020}
\bibentry{stowe_linguistic_2019}
\bibentry{moeller_customizing_2019}
\bibentry{moeller_improving_2019}
\bibentry{liu_morphological_2018}
\bibentry{moeller_neural_2018}
\bibentry{moeller_automatic_2018}
\bibentry{schneider_comprehensive_2018}
\bibentry{boerger2016language}
\bibentry{moeller_review_2014}
\bibentry{moeller_word_2011}
\bibentry{ndjerareou_brief_2011}


%%%% Presentations %%%% ----------------------------------------------------------------

%\NewPart{Presentations}{}

%\begin{hangparas}{.25in}{1}
%Sarah Moeller. 2015. “Developments in SayMore: The language documentation tool for citizen scientists” Paper presented at the 4th International Conference on Language Documentation and Conservation, University of Hawaii at Manoa, February 27-March 1, 2015.
%\sepspace

%Sarah Moeller. 2011. "Ups and Downs: Bringing Students and Language Communities Together" Poster presented at the 2nd International Conference on Language Documentation and Conservation, University of Hawaii at Manoa, February 11-13, 2011, and 3rd Dallas-Ft. Worth Metroplex Linguistics Conference, University of North Texas (UNT), September 10, 2011.
%\sepspace

%Sarah Moeller. 2010. “A preliminary analysis of Crimean Tatar written discourse.” Paper presented at 2nd Dallas-Ft. Worth Metroplex Linguistics Conference, November 5, 2010, Graduate Institute of Applied Linguistics.
%\sepspace
%\end{hangparas}

%%% Classes 
%%% ------------------------------------------------------------

\NewPart{Linguistics Classes/Workshops Taught}{}

\CourseEntry{Programming for Linguistics}{Instructor, University of Colorado}{2020}
%\CourseEntry{Database Design, Annotating, and Archiving}{Workshop, CoLang, University of Alaska}{2016}
%\CourseEntry{SayMore}{Workshop, Transarch, University of Bolzano}{2015}
%\CourseEntry{Introduction to Grammatical Analysis}{Instructor, Dallas Intl. University}{2015}
%\CourseEntry{Language Documentation}{Associate Instructor, Dallas Intl. University}{2011-2015}
%\sepspace

%\CourseEntry{Programming for Linguistics}{Teaching Assistant, University of Colorado}{2019}
%\CourseEntry{Languages of the World}{Teaching Assistant, University of Colorado}{2017-2019}
%\CourseEntry{Introduction to Linguistics}{Teaching Assistant, University of Colorado}{2016}
%\CourseEntry{Introduction to Grammatical Analysis}{Teaching Assistant, Dallas Intl. University}{2010}
%\CourseEntry{Second Language and Culture Acquisition}{Teaching Assistant, Dallas Intl. University}{2010}





\end{document}
